\chapter*{ \large ЗАКЛЮЧЕНИЕ}
\addcontentsline{toc}{chapter}{ЗАКЛЮЧЕНИЕ}
В ходе работы были исследованы новейшие принципы генерации изображений на основе автокодировщиков, а также особенный тип обучения моделей данного типа. Также были изучены основные принципы байесовской оптимизации и регрессии Гауссовского процесса.

Результатом работы стало обученный на датасете лиц знаменитостей celebA автокодировщик, способный генерировать лица несуществующих людей. Тип обучения сопровождался специальной пространственной функцией потери \\ (perceptrual loss). Также результатом стало некоторое приложение основанное на байесовской оптимизации, которое способно итеративно подбирать лицо по заданным характеристикам.

В перспективе ставятся задачи улучшения способа поиска оптимальных областей, значений у целевой функции на основе Байесовской оптимизации при помощи измения типа целевой фунции и способа оценивания изображений, а также улучшение качества работы модели автокодировщика путём усовершенствования архитектуры моделей и усложнения пространственной функции потери (perceptrual loss function).