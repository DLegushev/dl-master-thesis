Дипломная работа содержит:
\begin{itemize}
	\setlength{\itemindent}{2em}
	\setlength\itemsep{0.1em}
	\item 57 страниц,
	\item 16 иллюстарций,
	\item 3 приложений,
	\item 6 использованных источников.
\end{itemize}

\indent
Ключевые слова: МОДЕЛЬ СКРЫТЫХ ПЕРЕМЕННЫХ, EM-АЛГОРИТМ, ВАРИАЦИОННЫЕ АВТОКОДИРОВЩИКИ, ГАУССОВСКИЙ ПРОЦЕСС, БАЕСОВСКАЯ ОПТИМИЗАЦИЯ. \\
\indent
Объект исследования: Методы генарции изображения лица человека по заданным параметрам. \\
\indent
Цель исследования: Итеративное генерирование изображения лица человека по заданным параметрам. \\
\indent
Для достижения поставленной цели использовались:
\begin{itemize}
	\setlength{\itemindent}{2em}
	\setlength\itemsep{0.1em}
	\item Вариационный автокодировщик,
	\item Параметры генерации для автокодировщика задавались при помощи поиска оптимальных значений Гауссовского процесса на основе Байесовской оптимизации.
\end{itemize}

\indent
В дипломной работе получены следующие результаты:
\begin{enumerate}
	\setlength{\itemindent}{2em}
	\setlength\itemsep{0.1em}
	\item Изучена теория вариационных автокодировщиков, гауссовского процесса и байесовской оптимизации,
	\item Получена модель вариационного автокодировщика для генерации лиц,
	\item Реализован алгоритм работы Байесовской оптимизации,
	\item Создана утилита для интерактивной генерации лиц по заданным параметрам.  
\end{enumerate}

При разработке модели использовался язык программирования Python, библиотека для тензорных вычислений Tensorflow для реализации модели автокодировщика, библиотека GPy и GPyOpt для реализации нюансов алгоритма Байесовской оптимизации.

Дипломная работа является завершенной, поставленные задачи решены в полной мере, присутствует возможность дальнейшего развития исследований.

Дипломная работа выполнена автором самостоятельно.

\newpage

The thesis content:
\begin{itemize}
	\setlength{\itemindent}{2em}
	\setlength\itemsep{0.1em}
	\item 57 pages,
	\item 16 pictures,
	\item 3 appendixes.,
	\item 6 sources of information.
\end{itemize}

\indent
Keywords: LATENT VARIABLE MODEL, EM-ALGORITHM, VARIATIONAL AUTOENCODER, GAUSSIAN PROCESSING, BAYESIAN OPTIMIZATION. \\
\indent
Research object: Methods of generation person faces pictures by specified parameters. \\
\indent
Work purpose: Iterative generating picture of the person face by specified parameters. \\
\indent
To achieve the goal used:
\begin{itemize}
	\setlength{\itemindent}{2em}
	\setlength\itemsep{0.1em}
	\item The method of generation  was carried out using variational autoencoder (with Python and framework for tensor computing Tensorflow),
	\item Parameters of generation for autoencoder set by searching optimal values of Gaussian process on the base of Bayesian optimization(Python, library GPy and GPyOpt for nuance realization of Bayesian optimization algorithm).
\end{itemize}

\indent
In the thesis obtained the following results:
\begin{enumerate}
	\setlength{\itemindent}{2em}
	\setlength\itemsep{0.1em}
	\item Obtained the model of variational autoencoder for face generation. Bayesian optimization algorithm was implemented.
	\item Was created utility for interactive generation person faces by specified parameters.
\end{enumerate}

The thesis is completed, the tasks have been solved in full, there is the possibility of further development of research.

Thesis was done by the author himself.