\chapter*{\large РЕФЕРАТ}  
\addcontentsline{toc}{chapter}{РЕФЕРАТ}
В дипломной работе 57 страниц, 16 иллюстарций, 3 приложения, 6 использованных источников.

\indent
 МОДЕЛЬ СКРЫТЫХ ПЕРЕМЕННЫХ, EM-АЛГОРИТМ, ВАРИАЦИОННЫЕ АВТОКОДИРОВЩИКИ, ГАУССОВСКИЙ ПРОЦЕСС, БАЙЕСОВСКАЯ ОПТИМИЗАЦИЯ. \\
\indent
Объектом исследования дипломной работы являются методы генерации изображения лица человека по заданным параметрам. \\
\indent
Целью дипломной работы является реализация итеративного генерирование изображения лица человека по заданным параметрам. \\
\indent
Для достижения поставленной цели был использован язык программирования Python, библиотека для тензорных вычислений Tensorflow (метод генерации изображений на основе вариационного автокодировщика) и библиотеки GPy и GPyOpt для реализации нюансов алгоритма Байесовской оптимизации для подбора параметров генерации.

\indent
В дипломной работе получены следующие результаты:
\begin{enumerate}
	\setlength{\itemindent}{2em}
	\setlength\itemsep{0.1em}
	\item Модель вариационного автокодировщика для генерации лиц. Реализован алгоритм работы Байесовской оптимизации.
	\item Создана утилита для интерактивной генерации лиц по заданным параметрам.
\end{enumerate}

Дипломная работа является завершенной, поставленные задачи решены в полной мере, присутствует возможность дальнейшего развития исследований.

Дипломная работа выполнена автором самостоятельно. 

\newpage
\chapter*{\large РЭФЕРАТ} 
У дыпломнай рабоце 57 старонак, 16 малюнкаў, 3 дадатку, 6 істочникаў.

\indent
МАДЭЛЬ ПРЫХАВАНЫХ ПЕРАМЕННЫХ, EM- АЛГАРЫТМ, ВЫРАЯЦЫЙНЫЯ АЎТАКАДАВАЛЬНІК, ГАУСАЎСКІ ПРАЦЭС, БАЕСАЎСКАЯ АПТЫМІЗАЦЫЯ. \\
\indent
Аб'ектам даследавання дыпломная работы з'яўляецца метады згенеравання выявы асобы чалавека па зададзеных параметрах. \\
\indent
Мэтай дыпломная работы з'яўляецца ітэратыўнае генераванне выявы асобы чалавека па зададзеных параметрах. \\
\indent
Для дасягнення пастаўленай мэты выкарыстоўваўся язык прагамавання Python, бібліятэка для тэнзарных вылічэнняў Tensorflow (метад генерацыі малюнкаў на аснове варыяцыйнага автокодировщика) і бібліятэка GPy і GPyOpt для рэалізацыі нюансаў алгарытму Байесаўской аптымізацыі для падбору параметраў оптімізаціі.

\indent
У дыпломнай працы атрыманы наступныя вынікі:
\begin{enumerate}
	\setlength{\itemindent}{2em}
	\setlength\itemsep{0.1em}
	\item Мадэль варыяцыйнага аўтакадавальніка для генерацыі асоб. Рэалізаваны алгарытм працы Байесовской аптымізацыі.
	\item Створана ўтыліта для інтэрактыўнай генерацыі асоб па зададзеных параметрах.
\end{enumerate}

Дыпломная праца з'яўляецца завершанай, пастаўленыя задачы вырашаны ў поўнай меры, прысутнічае магчымасць далейшага развіцця даследаванняў.

Дыпломная праца выканана аўтарам самастойна.

\newpage
\chapter*{\large ABSTRACT} 
The thesis contains 57 pages, 16 pictures, 3 appendixes,6 sources of information.

\indent
LATENT VARIABLE MODEL, EM-ALGORITHM, VARIATIONAL \\ AUTOENCODER, GAUSSIAN PROCESSING, BAYESIAN OPTIMIZATION. \\
\indent
Research object is methods of generation person faces pictures by specified parameters. \\
\indent
Work purpose is iterative generating picture of the person face by specified parameters. \\
\indent
To achieve the goal used Python programming language, framework for tensor computing Tensorflow (the picture generation method on the base of variational autoencoders) and library GPy and GPyOpt for nuance realization of Bayesian optimization algorithm for selection parameters of generation.

\indent
In the thesis obtained the following results:
\begin{enumerate}
	\setlength{\itemindent}{2em}
	\setlength\itemsep{0.1em}
	\item Model of variational autoencoder for face generation. Bayesian optimization algorithm was implemented.
	\item Was created utility for interactive  generation person faces by specified \\ parameters.  
\end{enumerate}

The thesis is completed, the tasks have been solved in full, there is the possibility of further development of research.

Thesis was done by the author himself.

\newpage
