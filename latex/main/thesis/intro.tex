\chapter*{\large ВВЕДЕНИЕ}  
\addcontentsline{toc}{chapter}{ВВЕДЕНИЕ}
С развитием современных технологий и научного прогресса человечество начинает пытаться применять Байесовские методы во многих сферах жизнедеятельности, начиная с разработки компьютерных игр и заканчивая исследованием медицинских препаратов. Данный подход даёт ещё большие возможности многим алгоритмам машинного обучения: заполнение недостающих данных, извлечение гороздо большего количества информации из небольших датасетов, оптимизации параметров нейронных сетей и других моделей. Также Байесовские методы позволяют нам оценивать неопределённости в анализе данных, что является необходимой особенностью для таких областей, как медицина. Применительно к алгоритмам глубокого обучения, Байесовские методы позволят сжимать глубокие модели в сотни раз, автоматически подбирать гиперпараметры процесса обучения, сохраняя время и деньги.

Байесовская оптимизация - это подход к оптимизации целевых функций, для оценки которого требуется много времени (минут или часов). Данный процесс лучше всего подходит для оптимизации в непрерывных пространствах размерности менее 20 и допускает стохастический шум при вычислении функции. Он строит статистическую модель для целевой функции и количественно определяет неопределенность в этой моделе, используя регрессию гауссовского процесса и метод байесовского машинного обучения, а затем использует функцию оценки, определенную из этой модели, чтобы решить, какая наиболее оптимальная позицию, чтобы вычислить целевую функцию. 

В дипломной работе будет показано как работает байесовская оптимизация и регрессия гауссовского процесса на примере итеративной генерации лиц несуществующих людей по заданным характеристикам. При этом генерация будет осуществляться при помощи модели с набором нейросетей определённого типа, которые обучались на основе новейших исследований. 